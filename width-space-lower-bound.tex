%%%%%%%%%%%%%%%%%%%%%%%%%%%%%%%%%%%%%%%%%
% Beamer Presentation
% LaTeX Template
% Version 1.0 (10/11/12)
%
% This template has been downloaded from:
% http://www.LaTeXTemplates.com
%
% License:
% CC BY-NC-SA 3.0 (http://creativecommons.org/licenses/by-nc-sa/3.0/)
%
%%%%%%%%%%%%%%%%%%%%%%%%%%%%%%%%%%%%%%%%%

\documentclass[t,usenames,dvipsnames]{beamer}
\mode<presentation> {
\usetheme{Boadilla}
\setbeamertemplate{navigation symbols}{} % To remove the navigation symbols from the bottom of all slides uncomment this line
}

\usepackage{graphicx} % Allows including images
\usepackage{booktabs} % Allows the use of \toprule, \midrule and \bottomrule in tables
\usepackage{amsmath}
\usepackage{amsthm}
\usepackage{cancel}
\usepackage{bussproofs}
\usepackage{proof}

\newcommand {\nconf}[1]{\operatorname{neg}({#1})}

\title[space vs. width in resolution]
{Small space in resolution calculus implies small width
\texorpdfstring{\footnote{\tiny Filmus et al., From Small Space to Small Width in Resolution, ACM Trans.
Comput., 2015.}}}

\author{Narek Bojikian} % Your name
\institute[hu-berlin] % Your institution as it will appear on the bottom of every slide, may be shorthand to save space
{
Humboldt University of Berlin\\ % Your institution for the title page
\medskip
\textit{bojikian@informatik.hu-berlin.de} % Your email address
}
\date{10.07.2020} % Date, can be changed to a custom date

\begin{document}

\begin{frame}
\titlepage % Print the title page as the first slide
\end{frame}
\begin{frame} \frametitle{Introduction}
	\begin{itemize}[<+->]
		\item Literals, clauses, terms and normal forms.
			\begin{itemize}[<+->]
				\item Let $V = \{x, y, z\}$ be set of boolean-variables.
				\item Literals are negative and positive variables,
					\hspace{1cm} e.g. $x, \overline y,\dots$.
				\item A clause is a disjunction of literals,
					\hspace{1cm} e.g. $C = x \lor \overline y$.
				\item A term is conjunction literals,
					\hspace{1cm} e.g. $T = x \land \overline y$.
				\item A CNF formula is a conjunction of clauses.\\
					\hspace{1cm} e.g. $F := \{ \{x, y\}, \{\overline y, z\}\}$.
				\item A DNF
					\footnote{\uncover<8->{We write Clauses, terms and normal forms
					as sets and differentiate between them from the context,
					e.g. we write the clause $(x \lor \overline y)$ as $\{x,
					\overline y\}$}.} formula is a disjunction of terms.\\
				\item[]
				\item[--] A clause $C$ subsumes another clause $C'$ if $C \subseteq C'$.
				\item[--] We denote the empty clause by $\bot$ and the empty term
					by $\emptyset$.
				\item[--] A k-CNF (k-DNF) Formula is a CNF (DNF) formula where each
					clause (term) has at most $k$ literals.
			\end{itemize}
		\item Assignments.
			\begin{itemize}
				\item Boolean functions $\alpha: V \rightarrow \{0, 1\}$.
				\item $\alpha$ ``staisfies'' a circuit $F$ ( i.e. $\alpha \models F$), if
					$\alpha(F) = 1$ 
			\end{itemize}
	\end{itemize}
	
\end{frame}
\begin{frame}\frametitle{Introduction (Cont.) - Resolution proof}
	\begin{itemize}[<+->]
		\item A resolution configuration is a set of clauses.
		\item A resolution refutation of a CNF formula.
		\item[]
	\end{itemize}
	\uncover<3->{ \vspace{-.5cm}\begin{block}{Resolution refutation}
		A resolution refutation of a CNF-Formula $F$ is a sequence of configuration
		$\mathbb{C}_0 \dots \mathbb{C}_{\tau}$ such that:
		\begin{itemize}[<+->]
			\item $\mathbb{C}_0 = \emptyset$.
			\item $\bot \in \mathbb{C}_{\tau}$.
			\item $\mathbb{C}_i$ results from $\mathbb{C}_{i-1}$ by one of the following
				rules:
				\begin{itemize}
					\item[--] Axiom download. $\mathbb{C}_i = \mathbb{C}_{i-1} \cup
						\{A\}$, where $A \in F$ and $ A \notin
						\mathbb{C}_{i-1}$.
					\item[--] Erasure. $\mathbb{C}_i = \mathbb{C}_{i-1} \setminus
						\{A\}$, where $A \in \mathbb{C}_{i-1}$.
					\item[--] Inference. $\mathbb{C}_i = \mathbb{C}_{i-1} \cup
						\{A\}$, where $A \notin \mathbb{C}_{i-1}$ inferred
						by one of the following rules (here $\alpha$ denote
						a literal and $G, H$ denote clauses.
						\begin{itemize}
							\item
				$\displaystyle \frac{\overline{\alpha} \lor G \quad \alpha \lor H}{G \lor H}$
							\item 
				$\displaystyle \frac{G}{G\lor H}$ for an any clause $H$.
						\end{itemize}
				\end{itemize}
		\end{itemize}
	\end{block}}
\end{frame}

\begin{frame}\frametitle{Complexity measure of a resolution proof}
	\begin{block}{Complexity measures of a refutation}
		Let $\pi$ be a resolution refutation of a CNF formula $F$. We define the
		following complexity terms of $\pi$:
		\begin{itemize}[<+->]
			\item Length $L(\pi)$: the number of axiom-download and
				inference steps in $\pi$.
			\item Space $Sp(\pi)$: the maximum number of clauses in a
				configuration in $\pi$.
			\item Width $W(\pi)$: the size of the largest clause in $\pi$.
		\end{itemize}
		\uncover<4->{
		The length ($L(F \vdash \bot)$) of refuting $F$ is the minimum length of a
		refutation of $F$.}
		
		\uncover<5->{In an analogous way we define the space $Sp(F \vdash \bot)$ and the
		width $W(F \vdash \bot)$ of refuting $F$.}
	\end{block}
\end{frame}
\begin{frame}\frametitle{Example - Resolution refutation}
	$F = \{ \{a, b\}, \{\overline a\}, \{\overline b\}\}.$ \pause
	\begin{center}
	\begin{tabular}{p{.45\textwidth}|p{.45\textwidth}}
		Refutation  $(\pi)$ & Rule\\ 
		\midrule
		$\emptyset$ &\\
		$\{\{a, b\}\}$ & Axiom Download\\
		$\{\{a, b\}, \{\overline a\}\}$ & Axiom Download\\
		$\{\{a, b\}, \{\overline a\}, \{b\}\}$ & Inference\\
		$\{\{b\}\}$ & Erasure\\
		$\{\{b\}, \{\overline b\}$ & Axiom Download\\
		$\{\} = \bot$& Inf.+Er.\\
	  \end{tabular}
	\end{center} \pause
	\begin{itemize}[<+->]
		\item $L(\pi) = 5$.
		\item $Sp(\pi) = 3$.
		\item $W(\pi) = 2$.
	\end{itemize}
\end{frame}
\begin{frame}
	\frametitle{Space is as a lower-bonud for width}
	\begin{block} {Theorem~3.1 [cf. Atserias and Dalmau, 2008]}
		Let $F$ be a $k$-CNF formula and let $\pi:F\vdash \bot$ be a resolution refutation
		in clause space $Sp(\pi) = s$. Then, there is a resolution refutation $\pi'$ of $F$
		in width $W(\pi') \leq s + k - 3$.
	\end{block}
	Proof idea and example:

	\pause
	\only<2-5>{
		\begin{itemize}[<+->]
		\item  Negate each configuration in the refutation.
		\item  Rewrite resulting formulas as configurations.
		\item[--] Width is bounded by number of clauses in a configuration (space).
		\item The resulting sequence in \textbf{backward} order is (almost) a valid
			refutation.
	\end{itemize}}
	\only<6>{ $F = \{ \{a, b\}, \{\overline a\}, \{\overline b\}\}.$ \pause
	\begin{center}
	\begin{tabular}{p{.25\textwidth}|p{.2\textwidth}|p{.27\textwidth}}
		Refutation &  Negation & Neg. as a conf.\\ 
			\midrule
		$\emptyset$ & $\bot$ & $\bot$\\
		%
		$\{\{a, b\}\}$ &
		$\overline a \land \overline b$&
		$\{\{\overline a\}, \{\overline b\}\}$\\
		%
		$\{\{a, b\}, \{\overline a\}\}$&
		$(\overline a \land \overline b)\lor a$&
		$\{\cancel{\{\overline a, a\}}, \{\overline b, a\}\}$\\
		%
		$\{\{a, b\}, \{\overline a\}, \{b\}\}$& 
		$(\overline a \land \overline b) \lor a \lor \overline b$&
		$\{\cancel{\{\overline a, a, \overline b\}}, \{\overline b, a, \overline b\}\}$\\
		%
		$\{\{b\}\}$&
		$\overline b$&
		$\{\{\overline b\}\}$\\
		%
		$\{\{b\}, \{\overline b\}\}$&
		$\overline b \lor b$&
		$\{\cancel{\{\overline b, b\}}\} = \emptyset$\\
		%
		$\{\} = \bot$& $\emptyset$ & $\emptyset$\\
	\end{tabular}
	\end{center}}
\end{frame}

\begin{frame}\frametitle{Theorem~3.1 Proof sketch}
	\begin{block} {Negated configuration ($\nconf{\mathbb{C}}$)}
		$\nconf{\mathbb{C}}$ of a clause configuration $\mathbb{C}$ is recursively defined
		as follows:
		\vspace{-.3cm}
		\pause
		\begin{alignat*}{3}
			&\nconf{\emptyset} = \{\bot\}&&\\
			%
			\uncover<3->{
			&\nconf{\mathbb{C} \cup \{C\}} =&& \{\underbrace{D \lor \overline a | D \in
					\nconf{\mathbb{C}}; a \in C \setminus D }_{(*)};\\
			& &&\underbrace{\not \exists B \in \nconf{\mathbb{C}} s.t. B \lor \overline a \not
	\subseteq D \lor \overline a}_{(**)}\}.}
		\end{alignat*}
	\end{block}
	\pause 
	\pause
	\begin{itemize}[<+->]
		\item The part $(*)$ represents one clause of $\nconf{\mathbb{C}}$ as seen in the
			previous example (due to distributivity of $\lor$).
		\item The part $(**)$ ensures the minimality (inclusion wise) of the clauses and
			hence no trivial or subsummed clauses are contained in $\nconf{\mathbb{C}}$.
	\end{itemize}
\end{frame}

\begin{frame}\frametitle{Theorem~3.1. Proof sketch (Cont.)}
	\begin{block}{Observation~3.3.}
		The width of any clause in the negated configuration $\nconf{\mathbb{C}}$ is at most
		$Sp(\mathbb{C}_t) = |\mathbb{C}|$.
	\end{block}
	\pause
	\only<2>{ \noindent Proof sketch. Each clause in $\mathbb{C}$ contributes at most one literals to
	each clause inthe negated configuration.\hfill $\square$}
	\pause
	\begin{block}{Proposition~3.4.}
		The negated configuration $\nconf{\mathbb{C}}$ is the set of all minimal
		(nontrivial) clauses $C$ such that $\lnot C$ implies the configuration $\mathbb{C}$.
		That is,
		\vspace{-.3cm}
		$$\nconf{\mathbb{C}} = \{C|\lnot C \models \mathbb{C} \text{ and for every } C' \not
		\subseteq C \text{ it holds that } \lnot C' \not \models \mathbb{C}.\}$$
	\end{block}
	\pause
	\only<4>{ \noindent Proof Idea. Let $\mathbb{D}$ be the set of minimal clauses whose complements imply
	$\mathbb{C}$. Show that for $C \in \nconf{\mathbb{C}}$ there is a $C' \in \mathbb{D}$ such
	that $C' \subseteq C$ and vise versa. Since both sets contain only minimal clauses, the
	claim follows.}
	\pause
	\only<5->{ \begin{block}{Observatoin~3.5.}
		An assignment satisfies a clause configuration $\mathbb{C}$ if and only if it
		falsifies the negated clause configuration $\nconf{\mathbb{C}}$. That is,
		$\mathbb{C}$ is logically equivalent to $\lnot \nconf{\mathbb{C}}$.
	\end{block}}
\end{frame}
\begin{frame}\frametitle{Theorem 3.1. Proof sketch (Cont. 2)}
	\begin{block}{Lemma~3.6.}
		If $\mathbb{C} \models \mathbb{C}'$, then for every clause $C \in
		\nconf{\mathbb{C}}$ there exists a clause $C' \in \nconf{\mathbb{C}'}$ such that $C$
		is a weakening of $C'$.
	\end{block}
	\pause
	Remark. This ensures that erasure and inference do not increase the width over the maximum
	width of a clause in a refutation (i.e. at most $s$).
	\pause
	\begin{itemize}[<+->]
		\item Let $\pi = (\mathbb{C}_0, \mathbb{C}_1, \dots \mathbb{C}_{\tau})$.
		\item Let $\pi' = (\mathbb{D}_{\tau}, \dots, \mathbb{D}_0), \mathbb{D}_i =
			\nconf{\mathbb{C}_i}$.
		\item $\bot \in \mathbb{D}_0$ and $\mathbb{D}_{\tau} = \emptyset$.
		\item {\color{red}Assume} for now that $k \geq 3$.
		\item If $\mathbb{C}_{t-1} \models \mathbb{C}_{t}$ (Lemma~3.6.),
		\item[] \hspace{.5cm} $\implies$  $\mathbb{D}_{t-1}$ is a weakening of $\mathbb{D}_t$.
		\item Interesting case: $\mathbb{C}_{t-1} \not \models \mathbb{C}_{t}$
		\item[] \hspace{.5cm} ($\implies$ Axiom Download at step $t$).
	\end{itemize}
\end{frame}

\begin{frame}\frametitle{Theorem 3.1. Proof sketch (Cont. 3)}
	\begin{block}{Claim.}
		If $\mathbb{C}_t$ results from $\mathbb{C}_{t-1}$ through an axiom-download step, it
		follows that $W(\mathbb{D}_t \vdash \mathbb{D}_{t-1}) \leq s + k - 3$.
	\end{block}
	\pause
	\begin{minipage}[t]{0.6\linewidth}
	\begin{itemize}[<+->]
		\item Assume $Sp(\mathbb{C}_{t-1}) \leq s - 2$. \textbf{(Why?)}
		\item Let $\mathbb{C}_t = \mathbb{C}_{t-1} \cup A$,\\
			$A = \{a_1, \dots a_{\ell}\} \in F, \ell \leq k$.
		\item Choose $C \in \mathbb{D}_{t-1} \setminus \mathbb{D}_t$, (Obs. 3.3).
		\item[] $W(C) \leq Sp(\mathbb{C}_{t-1}) \leq s-2$ 
		\item First download $A$ then derive $C$\\
			from $\mathbb{D}_{t} \cup \{A\}$ as follows.
		\item Note that each $C \lor \overline a_i$ is\\
			in or a weakening of a clause in $\mathbb{C}_t$.
	\end{itemize}
	\end{minipage}%
	\hfill%
	\begin{minipage}[t]{0.39\linewidth}
	\pause
	$$
	\resizebox{\hsize}{!}{%
	\infer{C}{
	%
		\infer{\begin{aligned}C \lor a_3 \lor& \dots \lor a_{\ell}\\ &\vdots\\ C &\lor
		a_{\ell} \end{aligned}}
		{
			\infer{C \lor a_2 \lor \dots \lor a_{\ell}}
			{A & C\lor \overline a_1}
		     	&C \lor \overline a_2
		}
	%
	%
	& \begin{aligned}\vphantom{A}\\ \vphantom{\vdots}\\ C \lor \overline
	a_{\ell}\end{aligned}}%
	}
	$$
	\end{minipage}
\end{frame}

\begin{frame}\frametitle{Theorem 3.1. Proof sketch - Analysis}
	\noindent%
	\begin{minipage}[t]{.67\linewidth}
	\begin{itemize}[<+->]
		\item If $C = \emptyset$, the width bounded by $W(A) \leq k$.
		\item Else, the width is bounded by\\
			\hspace{1cm}$W(C) + W(A) - 1 \leq s + k - 3$.
	\end{itemize}
	\end{minipage}%
	\hfill%
	\begin{minipage}[t]{0.3\linewidth}
	$$
	\resizebox{\hsize}{!}{%
	\infer{C}{
	%
		\infer{\begin{aligned}C \lor a_3 \lor& \dots \lor a_{\ell}\\ &\vdots\\ C &\lor
		a_{\ell} \end{aligned}}
		{
			\infer{C \lor a_2 \lor \dots \lor a_{\ell}}
			{A & C\lor \overline a_1}
		     	&C \lor \overline a_2
		}
	%
	%
	& \begin{aligned}\vphantom{A}\\ \vphantom{\vdots}\\ C \lor \overline
	a_{\ell}\end{aligned}}%
	}
	$$
	\end{minipage}

	\begin{center}%
		\uncover<3->{What if {\color{red} $k < 3$}?}%
	\end{center}%
	\pause
	\uncover<4->{
		\begin{itemize}[<+->]
			\item $s \geq s + k - 3$. Problem!
			\item Remove all weakening steps (inference and erasure).
			\item Make sure that the resulting refutation $\pi''$ is is still valid.
			\item Width is now $\max \{k, s+k-3\} = s+k-3$.
		\end{itemize}
	}
\end{frame}

\begin{frame}\frametitle{Generalization to r-DNF resolution}
	define r-DNF resolution and introduce the extra rules
\end{frame}

\begin{frame}[c]
	\color{NavyBlue} \centering \Large \textbf{
	Lowerbounds\\on the space of a resolution-refuataion.}
\end{frame}

\begin{frame} \frametitle{T-Joins and Tseitin formulas}
	definitions
\end{frame}

\begin{frame} \frametitle{Tseitin formulas have large width}
	Complexity measure and configuration existence with intermediate measure

	result in lemma  as block
\end{frame}

\begin{frame} \frametitle{Tseitin formulas have large width (Cont.)}
	State theorem as a block and write down proof.
\end{frame}

\begin{frame} \frametitle{Conclusions}
polynomial proofs?	
\end{frame}

\end{document} 
