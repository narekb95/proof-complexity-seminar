%%%%%%%%%%%%%%%%%%%%%%%%%%%%%%%%%%%%%%%%%
% Beamer Presentation
% LaTeX Template
% Version 1.0 (10/11/12)
%
% This template has been downloaded from:
% http://www.LaTeXTemplates.com
%
% License:
% CC BY-NC-SA 3.0 (http://creativecommons.org/licenses/by-nc-sa/3.0/)
%
%%%%%%%%%%%%%%%%%%%%%%%%%%%%%%%%%%%%%%%%%

\documentclass[t,usenames,dvipsnames]{beamer}
\mode<presentation> {
\usetheme{Boadilla}
\setbeamertemplate{navigation symbols}{} % To remove the navigation symbols from the bottom of all slides uncomment this line
}

\usepackage{graphicx} % Allows including images
\usepackage{booktabs} % Allows the use of \toprule, \midrule and \bottomrule in tables
\usepackage{amsmath}
\usepackage{amsthm}

\title[space vs. width in resolution]
{Small space in resolution calculus implies small width
\texorpdfstring{\footnote{\tiny Filmus et al., From Small Space to Small Width in Resolution, ACM Trans.
Comput., 2015.}}}

\author{Narek Bojikian} % Your name
\institute[hu-berlin] % Your institution as it will appear on the bottom of every slide, may be shorthand to save space
{
Humboldt University of Berlin\\ % Your institution for the title page
\medskip
\textit{bojikian@informatik.hu-berlin.de} % Your email address
}
\date{10.07.2020} % Date, can be changed to a custom date

\begin{document}

\begin{frame}
\titlepage % Print the title page as the first slide
\end{frame}
\begin{frame} \frametitle{Introduction}
	\begin{itemize}[<+->]
		\item Literals, clauses, terms and normal forms.
			\begin{itemize}[<+->]
				\item A set of boolean-variables $V = \{x, y, z\}$.
				\item Literals are negative and positive variables,\\
					\hspace{1cm} eg. $x, \overline y$.
				\item A clause
					\footnote{Clauses and terms are usually written as sets of
						literals, when it is clear from the context which of
					the two we are dealing with}
					is an ``or''-gate over a set of literals,\\
					\hspace{1cm} 
					eg . $C = \{x, \overline y\}$.
				\item A term is an ``and''-gate over a set of literals.
				\item ``Conjunctive Normal Form'' (CNF)\\
				\item[]	\hspace{1cm}A set of clauses -
					representing an ``and''-gate over them,\\
					\hspace{1cm} eg. $F := \{ \{x, y\}, \{\overline y, z\}\}$.
				\item ``Disjunctive Normal Form'' (DNF)\\
				\item[]	\hspace{1cm}A set of terms -
					representing an ``or''-gate over them.
			\end{itemize}
		\item Assignments.
			\begin{itemize}
				\item Boolean functions $\alpha: V \rightarrow \{0, 1\}$.
				\item $\alpha$ ``staisfies'' a circuit $F$ ($\alpha \models F$), if
					$\alpha(F) = 1$ 
			\end{itemize}
	\end{itemize}
	
\end{frame}
\begin{frame}\frametitle{Introduction (Cont.)}
	define resolution, allowed rules, refutations, refutation width and space 
\end{frame}
%----------------------------------------------------------------------------------------
\begin{frame}
	\frametitle{Space is as a lower-bonud for width}
	define neg(C).

	state the theorem in a block and prove it (maximal two extra frames)

	Add a drawing
\end{frame}
\begin{frame}[c]
	\color{NavyBlue} \centering \Large \textbf{
	Lowerbounds\\on the space of a resolution-refuataion.}
\end{frame}
\begin{frame} \frametitle{T-Joins and Tseitin formulas}
	definitions
\end{frame}
\begin{frame} \frametitle{Tseitin formulas have large width}
	Complexity measure and intermediate

	result in lemma  as block
\end{frame}
\begin{frame} \frametitle{Tseitin formulas have large width (Cont.)}
	State theorem as a block and write down proof.
\end{frame}
\begin{frame} \frametitle{Conclusion}
	
\end{frame}

\end{document} 
