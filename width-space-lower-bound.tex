%%%%%%%%%%%%%%%%%%%%%%%%%%%%%%%%%%%%%%%%%
% Beamer Presentation
% LaTeX Template
% Version 1.0 (10/11/12)
%
% This template has been downloaded from:
% http://www.LaTeXTemplates.com
%
% License:
% CC BY-NC-SA 3.0 (http://creativecommons.org/licenses/by-nc-sa/3.0/)
%
%%%%%%%%%%%%%%%%%%%%%%%%%%%%%%%%%%%%%%%%%

\documentclass[t,usenames,dvipsnames]{beamer}
\mode<presentation> {
\usetheme{Boadilla}
\setbeamertemplate{navigation symbols}{} % To remove the navigation symbols from the bottom of all slides uncomment this line
}

\usepackage{graphicx} % Allows including images
\usepackage{booktabs} % Allows the use of \toprule, \midrule and \bottomrule in tables
\usepackage{amsmath}
\usepackage{amsthm}

\title[space vs. width in resolution]
{Small space in resolution calculus implies small width
\texorpdfstring{\footnote{\tiny Filmus et al., From Small Space to Small Width in Resolution, ACM Trans.
Comput., 2015.}}}

\author{Narek Bojikian} % Your name
\institute[hu-berlin] % Your institution as it will appear on the bottom of every slide, may be shorthand to save space
{
Humboldt University of Berlin\\ % Your institution for the title page
\medskip
\textit{bojikian@informatik.hu-berlin.de} % Your email address
}
\date{10.07.2020} % Date, can be changed to a custom date

\begin{document}

\begin{frame}
\titlepage % Print the title page as the first slide
\end{frame}
\begin{frame} \frametitle{Introduction}
	\begin{itemize}[<+->]
		\item Literals, clauses, terms and normal forms.
			\begin{itemize}[<+->]
				\item Let $V = \{x, y, z\}$ be set of boolean-variables.
				\item Literals are negative and positive variables,
					\hspace{1cm} e.g. $x, \overline y,\dots$.
				\item A clause is a disjunction of literals,
					\hspace{1cm} e.g. $C = x \lor \overline y$.
				\item A term is conjunction literals,
					\hspace{1cm} e.g. $T = x \land \overline y$.
				\item A CNF formula is a conjunction of clauses.\\
					\hspace{1cm} e.g. $F := \{ \{x, y\}, \{\overline y, z\}\}$.
				\item A DNF
					\footnote{\uncover<8->{We write Clauses, terms and normal forms
					as sets and differentiate between them from the context,
					e.g. we write the clause $(x \lor \overline y)$ as $\{x,
					\overline y\}$}.} formula is a disjunction of terms.\\
				\item[]
				\item[--] A clause $C$ subsumes another clause $C'$ if $C \subseteq C'$.
				\item[--] We denote the empty clause by $\bot$ and the empty term
					by $\emptyset$.
			\end{itemize}
		\item Assignments.
			\begin{itemize}
				\item Boolean functions $\alpha: V \rightarrow \{0, 1\}$.
				\item $\alpha$ ``staisfies'' a circuit $F$ ( i.e. $\alpha \models F$), if
					$\alpha(F) = 1$ 
			\end{itemize}
	\end{itemize}
	
\end{frame}
\begin{frame}\frametitle{Introduction (Cont.) - Resolution proof}
	\begin{itemize}[<+->]
		\item A resolution configuration is a set of clauses.
		\item A resolution refutation of a CNF formula.
		\item Complexity measures of refuting formulas.
	\end{itemize}
	\only<2>{
		\vspace{-.5cm}
		\begin{block}{Resolution refutation}
		A resolution refutation of a CNF-Formula $F$ is a sequence of configuration
		$\mathbb{C}_0 \dots \mathbb{C}_{\tau}$ such that:
		\begin{itemize}
			\item $\mathbb{C}_0 = \emptyset$.
			\item $\bot \in \mathbb{C}_{\tau}$.
			\item $\mathbb{C}_i$ results from $\mathbb{C}_{i-1}$ by one of the following
				rules:
				\begin{itemize}
					\item[--] Axiom download. $\mathbb{C}_i = \mathbb{C}_{i-1} \cup
						\{A\}$, where $A \in F$ and $ A \notin
						\mathbb{C}_{i-1}$.
					\item[--] Erasure. $\mathbb{C}_i = \mathbb{C}_{i-1} \setminus
						\{A\}$, where $A \in \mathbb{C}_{i-1}$.
					\item[--] Inference. $\mathbb{C}_i = \mathbb{C}_{i-1} \cup
						\{A\}$, where $A \notin \mathbb{C}_{i-1}$ inferred
						by one of the following rules (here $\alpha$ denote
						a literal and $G, H$ denote clauses.
						\begin{itemize}
							\item
				$\displaystyle \frac{\overline{\alpha} \lor G \quad \alpha \lor H}{G \lor H}$
							\item 
				$\displaystyle \frac{G}{G\lor H}$ for an any clause $H$.
						\end{itemize}
				\end{itemize}
		\end{itemize}
	\end{block}}
	\only<3-7>{
		\begin{block}{Complexity measures of a refutation}
			Let $\pi$ be a resolution refutation of a CNF formula $F$. We define the
			following complexity terms of $\pi$:
			\begin{itemize}[<+->]
				\item Length $L(\pi)$: the number of axiom-download and
					inference steps in $\pi$.
				\item Space $Sp(\pi)$: the maximum number of clauses in a
					configuration in $\pi$.
				\item Width $W(\pi)$: the size of the largest clause in $\pi$.
			\end{itemize}
			\uncover<7>{
			The length ($L(F \vdash \bot)$) of refuting $F$ is the minimum length of a 
			refutation of $F$. In an analogous way we define the space $Sp(F \vdash
		\bot)$ and the width $W(F \vdash \bot)$ of refuting $F$.}
		\end{block}
	}
\end{frame}

\begin{frame}
	\frametitle{Space is as a lower-bonud for width}
	define neg(C).

	state the theorem in a block and prove it (maximal two extra frames)

	Add a drawing
\end{frame}
\begin{frame}[c]
	\color{NavyBlue} \centering \Large \textbf{
	Lowerbounds\\on the space of a resolution-refuataion.}
\end{frame}
\begin{frame} \frametitle{T-Joins and Tseitin formulas}
	definitions
\end{frame}
\begin{frame} \frametitle{Tseitin formulas have large width}
	Complexity measure and intermediate

	result in lemma  as block
\end{frame}
\begin{frame} \frametitle{Tseitin formulas have large width (Cont.)}
	State theorem as a block and write down proof.
\end{frame}
\begin{frame} \frametitle{Conclusion}
	
\end{frame}

\end{document} 
