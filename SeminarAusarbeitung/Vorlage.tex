% Vorlage f�r Seminar-Ausarbeitungen
%
% Dateiname: Vorlage.tex
% zuletzt ge�ndert: 11. Februar 2016
% Autorin: Nicole Schweikardt

% Definition der Dokument-Klasse, in diesem Falle die Klasse SeminarAusarbeitung,
% die in der Datei SeminarAusarbeitung.cls bereitgestellt wird
\documentclass{SeminarAusarbeitung}

\title{Clause space vs. width in resolution}
\author{Narek Bojikian}
\seminar{Beweiskomplexit�t}
\semester{Sommersemester 2020}
\leitung{Prof.\ Dr.\ Christoph Berkholz} 
\institut{Institut f�r Informatik}
\universitaet{Humboldt-Universit�t zu Berlin}

\usepackage{bbold}
\usepackage{xcolor}
\usepackage{graphicx}
\usepackage{float}
\usepackage{natbib}
\usepackage{caption}
\usepackage{tikz}
\usepackage{bussproofs}
\usepackage{proof}

\newcommand{\conf}{\mathbb{C}}
\newcommand{\negc}[1]{\operatorname{neg}({#1})}
\newcommand{\negd}[1]{\operatorname{neg}_{\mathcal R(d)}({#1})}
\newcommand{\prty}{\operatorname{PARITY}}

\begin{document}

\maketitle
\begin{abstract}
	In this report we summarize our presentation of the paper "From Small Space to Small Width
	in Resolution" by Yuval Filmus, et al. \cite{DBLP:journals/tocl/FilmusLMNV15}. We focus on
	the first part of the paper, where the authors provide a constructive proof that resolution
	refutations of low clause space imply refutations of low width. We end with a brief
	go-through of the results in the second part of the paper.
\end{abstract}

\section{Introduction}
Proof systems have been prominent in computer science, as a tool to solve the satisfiability problem
and as a method to prove unsatisfiability of boolean formulas. One of the most popular proof systems
is resolution. It refutes CNF-formulas and it is the base of the DPLL-Algorithm. It is well-studied
in theoretical computer science. For example, exponential lower-bounds have been proved for the
length of resolution refutations of the pigeonhole principle \cite{DBLP:journals/tcs/Haken85}. Along
side the length of the proof, other complexity measurements of such proofs have been introduced and
well studied. Among others are width, depth and clause space.

In the paper of our interest, the authors show that a refutation of low clause space yields a
refutation of low width. A result that had been proven by A. Atserias and V. Dalmau in 2003
\cite{DBLP:journals/jcss/AtseriasD08}. In the current paper however, the authors present a
constructive proof by manipulating a given refutation of low space to achieve a refutation of low
width. They also generalize the proof to the $r$-DNF proof system. In the second part of the paper,
the authors use this method to prove lower-bounds on the space of refuting Tseitin formulas -a class
of formulas that we define later in this report- using well-known lower-bounds on the width of these
formulas. As in the presentation, we focus on the first part of the paper presenting the techniques
developed by the authors. We finish by a short summary of the results of the second part.  On a side
note, the authors also present in the third part of the paper an analysis of the polynomial calculus
proof system, where they show that similar results are not probably easy to achieve in this system.
We skip this part in our summary, since no major results were proven there, and suffice to mention
the intuition behind it at the end.

\section{Preliminaries and notation}
We begin this section with formal definitions of formulas, normal forms and assignments. We follow
that by defining the previously mentioned proof systems.
\subsection{Boolean formulas and assignments}
\begin{itemize}
	\item A Boolean formula $\varphi$ is a Boolean combination of a given
		set of variables $\operatorname{Var}(\varphi) := x_1, \dots, x_n$ and
		constants $\mathbb{0}$ and $\mathbb{1}$.
	\item An assignment $\alpha$ of $\varphi$ is a Boolean function that assigns to
		each variable of the formula a Boolean value
		$$\alpha\operatorname{Var}(\varphi) \rightarrow \{\mathbb{0,
		1}\}.$$ We write $\alpha \models \varphi$ for $\alpha$ satisfies
		$\varphi$.
	\item An assignment $\alpha$ satisfies a given formula $\varphi$ if and only
		if $[\varphi]^{\alpha} = \mathbb{1}$.
	\item A formula is in the negation normal form (\textbf{NNF}), if it
		contains only disjunctions and conjunctions of positive or
		negative literals.
	\item A formula is in the conjunctive normal form (\textbf{CNF}), if it
		is a conjunction of one or more clauses, where each clause is a
		disjunction of one or more literals.
	\item[-] A $d$-CNF formula is a CNF-formula, where each clause contains at most $d$ literals.
	\item A formula is in the disjunctive normal form (\textbf{DNF}), if it
		is a conjunction of one or more clauses, where each clause is a
		disjunction of one or more literals.
	\item[-] A $d$-DNF formula is a DNF-formula, where each term contains at most $d$ literals.
	\item Given a formula $\varphi$ along with an assignment $\alpha$ for
		$\varphi$. Let $V \subseteq \operatorname{Var} (\varphi)$ be a
		subset of variables. We define the partial assignment
		$\alpha_{|V}$ as the restriction of $\alpha$ to $V$, i.e.
		$$\alpha_{|V}:V\rightarrow \{\mathbb{0}, \mathbb{1}\}:x\mapsto
		\alpha(x).$$
		
	\item[-] Given a CNF-formula $\varphi$ and a partial assignment $\alpha_{|V}$, we say that
		$\alpha_{|V}$ satisfies $\varphi$ if for each clause $C$ in $\varphi$, there is a
		literal $a \in C$ where $\alpha_{|V} \models a$.
\end{itemize}


\subsection{Resolution proof system and r-DNF resolution}
Now we formally introduce resolution refutation. We call a set of clauses a configuration. Given a
CNF-formula $F$, a refutation of $F$ is a sequence of configurations $\conf_0, \dots \conf_{\tau}$
such that
\begin{itemize}
	\item $\conf_0 = \emptyset$.
	\item $\bot \in \conf_{\tau}$.
	\item $\conf_i$ results from $\conf_{i-1}$ by one of the following rules:
		\begin{itemize}
			\item Axiom downlaod. $\conf_i = \conf_{i-1} \cup \{A\}; A \in F \setminus
				\conf_{i-1}$.
			\item Erasure. $\conf_i = \conf_{i-1} \setminus \{A\}; A \in \conf_{i-1}$.
			\item Inference. $\conf_i = \conf_{i-1} \cup \{A\}$, where $A \notin
				\conf_{i-1}$ inferred by one of the following rules:
				\begin{enumerate}
					\item $\infer{G\lor H}{\overline{\alpha}\lor G & \alpha \lor
						H}$, where $G$ and $H$ are two arbitrary clauses.
					\item $\infer{G\lor H}{G}$, for an aribrary clause $G$ and
						any clause $H$.
				\end{enumerate}
		\end{itemize}
\end{itemize}
Intuitively, configurations along a refutation are sets of clauses implied by the given CNF-formula
that we use to refute it. We use erasure to keep the clause space of a refutation bounded, and
hence, in an optimal refutation a configuration is the set of all clauses derived until the step $i$
that will be needed in steps later than $i$. It is not hard to see that the resolution proof system
is correct and complete.

Now that we have introduced resolution, we list the most common complexity measures of a resolution
proof. We have already mentioned these measures above and we shall use them to present the results
later in this report. Let $\pi$ be a given refutation of a formula $F$. First we define the
\textbf{Length} of the refutation $L(\pi)$ as the number of non-erasure steps in the refutation. The
\textbf{Width} of a refutation $W(\pi)$ is the maximum number of literals in a clause in the
refutation. The \textbf{clause space} $Sp(\pi)$ of the refutation $\pi$ is the largest configuration
along $\pi$. We define each of the previous measures over a CNF-formula as the minimum of this
measure over all refutations of this formula or $\infty$ if the formula is satisfiable.

We define the $d$-DNF resolution as a generalization of resolution, where we are given a CNF-formula
and the configurations are allowed to be sets $k$-DNF-formulas for $k \leq d$ instead of sets of
clauses.  Note that a clause is a $1$-DNF formula, and hence, $d$-DNF is a generalization of
resolution. In order to get use of the relaxation we define the following four inference rules
instead of the previous two in resolution: Let $G, G'$ and $H$ be arbitrary $d$-DNF formulas and $T,
T'$ arbitrary terms.
	\begin{enumerate}
		\item r-cut. $\infer{G\lor H}{(a_1\land\dots \land a_{r'})\lor G
			&\overline{a}_1 \lor \dots \lor \overline{a}_{r'} \lor
			H}$, where $r'\leq r$.
		\item Weakening. $\infer{G\lor H}{G}$, for any $r$-DNF $H$.
		\item $\land$-introduction. $\infer{G \lor (T \land T')}{G\lor T &G \lor T'}$, for
			$|T\cup T'| \leq r$.
		 \item $\land$-elimination. $\infer{G \lor T'}{G\lor T}$, for nonempty
			 $T' \subseteq T$.
	\end{enumerate}

	Clearly r-cut correspond to the first resolution rule and weakening to the second one,
	meanwhile $\land$-introduction and $\land$-elimination make use of the fact that we allow
	terms here and are not relevant in the resolution case.

\section{Width as a lower bound for space}
Now we are ready to preset the main results in the paper. We begin with the main theorem. We flow it
with a couple of auxiliary lemmas and then sketch the proof of the theorem. We generalize the lemma
after that to span the $d$-DNF resolution and show how to generalize the proof to span this case.
\begin{theorem}[Theorem~3.1 cf. Asterias and Dalmau, 2008 \cite{DBLP:journals/jcss/AtseriasD08}]
	\label{theo:main1}

	Let $F$ be a $k$-CNF formula and let $\pi : F \vdash \bot$ be a resolution refutation in
	clause space $Sp(\pi) = s$. Then, there is a resolution refutation $\pi'$ of $F$ in width
	$W(\pi') \leq s + k - 3$.
\end{theorem}
Intuitively, we derive $\pi'$ from $\pi$ by negating all configurations, and writing them back as
sets of clause in the reverse order. We bound the width of the resulting clauses by the number of
clauses in a configuration in $\pi$. We show that the resulting derivation is almost valid and we
show how to make it a valid derivation with increasing the width more than $k - 3$.

In order to preset the proof formally, we start by defining the negation of a configuration. We
define $\negc{\mathbb{C}}$ of a clause configuration $\mathbb{C}$ recursively as follows:
\begin{itemize}
	\item $\negc{\emptyset} = \{\bot\}$.
	\item $ \negc{\mathbb{C} \cup \{C\}} = \{\underbrace{D \lor \overline a | D \in
			\negc{\mathbb{C}}; a \in C \setminus D }_{(*)};
		\underbrace{\not \exists B \in \negc{\mathbb{C}} s.t. B \lor \overline
			a \varsubsetneq D \lor \overline a}_{(**)}\},$
\end{itemize}
where $(*)$ says that a clause is in $\negc{\mathbb{C}}$ if its negation implies the clauses in
$\mathbb{C}$ (Proposition~\ref{prop:neg}), and $(**)$ ensures the minimality of the clauses in
$\negc{\mathbb{C}}$ by ignoring subsumed clauses.

\begin{observation}[Obersvation 3.3.]\label{obs:width}
	The width of any clause in the negated configuration $\negc{\mathbb{C}}$ is at most
	$Sp(\mathbb{C}) = |\mathbb{C}|$.
\end{observation}
In order to see this, note that each clause in $\mathbb{C}$ contributes at most one literal to a
clause in the negated configuration.$\hfill \square$

\begin{proposition}[Proposition 3.4.]\label{prop:neg}
	The negtaed configuration $\negc{\mathbb{C}}$ is the set of all minimal (nontrivial clauses
	$C$ such that $\lnot C$ implies the configuration $\mathbb{C}$.
\end{proposition}
\noindent\underline{Proof idea.} Set $\mathbb{D}$ to be the set of all minimal clauses whose
complements imply $\mathbb{C}$. Show that for each clause $C \in \negc{\mathbb{C}}$ there is a
clause in $\mathbb{D}$ that subsumes $C$ and vise versa. By the minimality of both sets we get
$\mathbb{D} = \negc{\mathbb{C}}$ \hfill $\square$.


\begin{observation}[Observation 3.5.]
	An assignment satisfies a clause configuration $\mathbb{C}$ if and only if it falsifies the
	negated clause configuration $\negc{\mathbb{C}}$. That is, $\mathbb{C}$ is logically
	equivalent to $\lnot\negc{\mathbb{C}}$.\hfill$\square$
\end{observation}

Now we use the following lemma to show that the negations of the configuratoins in a refutation $\pi$
in the reverse order build an almost valid refutation. Namely, the lemma covers inference and
deletion steps.
\begin{lemma}\label{lem:weakening}
	If $\mathbb{C} \models \mathbb{C}'$, then for every clause $C \in \negc{\mathbb{C}}$ there
	exists a clause $C' \in \negc{\mathbb{C}'}$ such that $C$ is a weakening of $C'$.
	\hfill$\square$
\end{lemma}

\noindent\underline{Proof sketch. (Theorem~\ref{theo:main1})}Now we are ready to sketch the proof of
Theorem~\ref{theo:main1}. Given a refutation $\pi = (\mathbb{C}_0, \dots, \mathbb{C}_{\tau})$ of
clause space at most $s$. Let $\pi' := (\mathbb{D}_{\tau}, \dots, \mathbb{D}_0)$, where
$\mathbb{D}_i = \negc{\mathbb{C}_i}$ for $i \in \mathbb{N}_{\leq \tau}$. By the definition of
$\operatorname{neg}$ we know that $\bot \in \mathbb{D}_0$ and $\mathbb{D}_{\tau} = \emptyset$.
Hence, it suffices to proof that each configuration is a valid result of the preceding one. If
$\mathbb{C}_{t-1} \models \mathbb{C}_t$, by the previous lemma we know that each clause in
$\mathbb{D}_{t-1}$ is a weakening of a clause in $\mathbb{D}_{t}$ and hence $\mathbb{D}_{t-1}$
follows from $\mathbb{D}_t$ through weakening and erasures with does not increase the width of the
refutation above the width of the clauses in both configurations. Since this covers inference and
erase steps (resulting configurations follow from the original one), the only case we still have to
inspect is axiom-download. First assume that $k \geq 3$. In the following claim, we assume that
$\mathbb{C}_t$ results from $\mathbb{C}_{t-1}$ through an axiom-download and show that we can derive
$\mathbb{D}_{t-1}$ from $\mathbb{D}_t$ without increasing the width over $s + k - 3$.

\begin{claim}
	If $\mathbb{C}_t$ results from $\mathbb{C}_{t-1}$ through an axiom-download step, it follows
	that $W(\mathbb{D}_t \vdash \mathbb{D}_{t-1}) \leq s + k - 3$.
\end{claim}
\noindent\underline{Proof sketch.} First we can assume that before each axiom-download we have $Sp(\mathbb{C}_{t-1}) \leq s-2$, since
after the axiom download the width will increase at least by one and if it was $s-1$, either the
same clause or another clause will get removed directly after the download. In the former case we
skip the whole step meanwhile in the other we do the erasure step before the axiom-download.

Let $\mathbb{C}_t = \mathbb{C}_{t-1} \cup A; A = \{a_1, \dots, a_{\ell}\} \in F, \ell \leq k$.
Choose $C \in \mathbb{D}_{t-1} \setminus \mathbb{D}_t$. Due to Observation~\ref{obs:width} we have
$W(C) \leq Sp(\mathbb{C}_{t-1}) \leq s-2$. First download $A$ then derive $C$ from $\mathbb{D}_t
\cup \{A\}$ as follows: (note that each $C\lor \overline a_i$ is in or a weakening of a clause in
$\mathbb{D}_t$)
$$
\infer{C}{
%
	\infer{\begin{aligned}C \lor a_3 \lor& \dots \lor a_{\ell}\\ &\vdots\\ C &\lor
	a_{\ell} \end{aligned}}
	{
		\infer{C \lor a_2 \lor \dots \lor a_{\ell}}
		{A & C\lor \overline a_1}
		&C \lor \overline a_2
	}
%
%
& \begin{aligned}\vphantom{A}\\ \vphantom{\vdots}\\ C \lor \overline
a_{\ell}\end{aligned}}%
$$
Now if $C = \emptyset$, then the width is bounded by $W(A) \leq k$. Else, the width is bounded by
$W(C) + W(A) - 1 \leq s + k - 3$. \hfill $\square$
Now if $k \leq 3$ we have $s\geq s + k - 3$. However, in this case it suffices to remove all
weakening steps in the derivation and still get a valid derivation. In this case we can bound the
width in $\max\{k, s + k - 3\} = s + k - 3$. \hfill $\square$

In the following theorem, we generalize the previous result to $d$-DNF resolution.
\begin{theorem}[Theorem~3.7.]\label{theo:ddnf}
	Let $F$ be a $k$-CNF formula and $\pi : F \vdash \bot$ be a $d$-DNF resolution refutation
	of $F$ in space $Sp(\pi) \leq s$. Then there exists a resolution refutation $\pi'$ of $F$ in
	width at most $W(\pi') \leq (s-2)d + k - 1$.
\end{theorem}
First we define the negation of a $d$-DNF configuration as follows
\begin{itemize}
	\item $\negd{a} = \{\bot\}$.
	\item $\negd{\mathbb{C}\cup \{C\}} = \{D \lor \overline T| D \in \negd{\mathbb{C}}; T \in
		C; \not\exists B \in \negd{\mathbb{C}} \text{ s.t. } B \lor
	\overline T \varsubsetneq D \lor \overline T ; D\lor \overline T \text{ nontrivial}\}$.
\end{itemize}
Considering terms as literals, $\negd{\mathbb{C}}$ is the set of all minimal clauses $D$ such that
$\lnot D \models \mathbb{C}$.

\noindent\underline{Proof idea (Theorem~\ref{theo:ddnf})} An $d$-DNF configuration of space $s$ can
be easily transformed into a resolution configuration of width $s\cdot d$. Hence, all the previous
results can be proved again for $d$-DNF resolution under a factor $r$ increase in the bounds. Axiom
downloads can be hence handled in the same way and inference and erasure steps follow in a similar
way to Lemma~\ref{lem:weakening}. In total we get a bound of $(s-2)d + k - 1$, where again in case
of $k < 2r + 1$ additional pruning is required as well.\hfill $\square$

\section{Lower-bounds on clause-space}
In this section, the authors use the previous results to provide a static technique to prove space
lower-bounds. They use this technique to prove a space lower-bound on refuting Tseitin formulas. In
order to achieve this, they introduce a new complexity measure and show that any refutation of
Tseitin a well-chosen Tseitin formula contains a configuration of intermediate complexity. They
follow that by proving, that such a configuration must have a large clause-space. We begin by
introducing Tseitin formulas.
\begin{definition}[Tseitin formula]
	Let $G = (V, E)$ be an undirected graph and $\chi : V \rightarrow \{0, 1\}$ be a function. Let
	us identify every edge $e \in E$ with a variable $x_e$, and let us write $\prty_{v, \chi}$
	to denote the canonical CNF encoding of the constrain $\sum_{e\ni v}x_e = \chi(v)
	\quad(\operatorname{mod} 2)$ for any vertex $v \in V$. Then the Tseitin formula over $G$
	with respect to $\chi$ is $Ts(G, \chi) = \bigwedge_{v \in V}\prty_{v, \chi}$.
\end{definition}

Note that $\prty_{v, \chi}$ has at most $2^{|\delta(x)|-1} \leq 2^{d-1}$ many clauses, for $d$ an
upper-bound over the degrees of the vertices in $G$. For $U \subseteq V$, let us say $U$ has even
(odd) charge, if $\sum_{u \in U} \chi(u)$ is even (odd). Since the sum of the degrees of the
vertices in a graph is even, $Ts(G, \chi)$ is clearly unsatisfiable if $V$ has odd charge. On the
other hand, if $V$ has even charge then $Ts(G, \chi)$ is always satisfiable.

Now we introduce the expansion property of graphs, depending on which we show the hardness of
Tseitin formulas.
\begin{definition}[Edge expander]
	The graph $G = (V, E)$ is an $(s, \delta)$-edge expander if for every set of vertices $U
	\subseteq V$ such that $|U| leq s$ it holds that $|\partial(U)| \geq \delta|U|$.
\end{definition}
Now we preset the main theorem of this section.
\begin{theorem}\label{theo:tseitin}
	For a Tseitin formula $Ts(G, \chi)$ over a $d$-regular $(s, \delta)$-edge epxander $G$ it
	holds that $Sp(Ts(G,\chi)) \geq \delta s/d$.
\end{theorem}

In order to sketch the proof, we need one more definition, namely the complexity measure introduced
by the authors.
\begin{definition}[Configuratoin complexity measure]
	The term complexity measure $\nu(T)$ of a term $T$ is $\nu(T) = \min\{|V'|:V' \subseteq V
	\text{ and } T \land \bigwedge_{v \in V'} \prty_{v, \chi} \models \bot\}$.
	
	The configuration complexity measure $\mu(\mathbb{C})$ of a resolution configuration
	$\mathbb{C}$ is defined as $\mu(\mathbb{C}) = \max\{\nu(T): T \models \mathbb{C}\}$. When
	$\mathbb{C}$ is contradictory we have $\mu(\mathbb{C}) = 0$.
\end{definition}
Clearly, $\nu$ is a monotone decreasing function, since $T \subseteq T'$ implies $\nu(T) \geq
\nu(T')$. Hence, in order to determine $\mu(\mathbb{C})$ it suffices to consider minimal terms $T$
for which $T \models \mathbb{C}$. These minimal terms are the negations of the terms in
$\negc{\mathbb{C}}$ as seen in the Proposition~\ref{prop:neg}.
\begin{definition}[Witness of measure.]
	A witness of the measure $\nu(T)$ of the term $T$ is a set of vertices $V^*$ for which
	$\nu(T) = |V^*|$ and $T \land \bigwedge_{v \in V^*}\prty_{v, \chi} \models \bot$. Similarly,
	for configurations $\mathbb{C}$ a witness for $\mu(\mathbb{C})$ is a term $T^*$ for which
	$\mu(\mathbb{C}) = \nu(T^*)$ and $T^* \models \mathbb{C}$.
\end{definition}

Now we present a sequence of results without stating the proofs. The proofs follow from the
definition of witnesses and complexity measures. We refer to the original papers for
more details. We follow the lemmas with a corollary that builds the main part of the proof the
previous theorem.
\begin{lemma}
	If $\mathbb{C} \models \mathbb{C}'$, then $\mu(\mathbb{C} \leq \mu(\mathbb{C}')$.
	\hfill $\square$
\end{lemma}
\begin{lemma}
	For a claues $A$ in $Ts(G, \chi)$ and a graph $G$ of bounded degree $d$, if $\mathbb{C}' =
	\mathbb{C} \cup \{A\}$, then $d\cdot \mu(\mathbb{C}') \geq \mu(\mathbb{C})$.
	\hfill $\square$
\end{lemma}
\begin{corollary}\label{cor:intermecdiatecomplexity}
	For any resolution refutation $\pi$ of a Tseitin formula $Ts(G, \chi)$ over a connected
	graph $G$ of bounded degree $d$ and any positive integer $r \leq |V|$ there exists a
	configuration $\mathbb{C} \in \pi$ such that the configuration complexity measure is bounded
	by $r/d \leq \mu(\mathbb{C} \leq r$.
\end{corollary}
The corollary follows from the previous lemma and the fact that there is a big gap between the
measure of the initial and final configurations, since we start with a configuration of measure
$|V|$ and reach a configuration of measure $0$ at the end. The previous lemma says that the measure
drops by factor at most $d$ in each step which implies that corollary. Now that we proved that every
resolution refutatoin of the Tseitin formula has a configuration of intermediate complexity, it
remains to show that a configuration having intermediate measure must also have large space. this
part of the proof relies on the expander property.

\begin{lemma}\label{lem:expanderspace}
	Let $G$ be an $(s, \delta)$-edge expander graph. For every configuration $\mathbb{C}$
	satisfying $\mu(\mathbb{C}) \leq s$ it holds that $Sp(\mathbb{C}) \geq \delta \cdot
	\mu(\mathbb{C})$.\hfill $\square$
\end{lemma}
Coroolary~\ref{cor:intermecdiatecomplexity} together with Lemma~\ref{lem:expanderspace} imply
directly Theorem~\ref{theo:tseitin}.\hfill $\square$ (choose $r = s$ for the corollary).

\section{Conclusion}
In this report we have seen, that the width of refuting a CNF-formula is a lower-bound on the
clause-space of refuting it under resolution. The result was generalized to $d$-DNF resolution. The
authors inspect the same bound in polynomial calculus. They show, that for pebbling formulas,
manipulating the refutation as seen in the report yields a duality between top-down and bottom up
refutations of these formulas. Meanwhile this duality is valid for top-down bounded space and bottom up
bounded degree refutations, there exists a bottom-up constant space refutation in polynomial calculus
meanwhile it seems unlikely, that a constant width top-down refutation exists as the authors argue.
Hence, such a duality might not exist in polynomial calculus and the approach might fail when
applied to polynomial calculus.







\bibliographystyle{plain}
\bibliography{ref.bib}
\end{document}
