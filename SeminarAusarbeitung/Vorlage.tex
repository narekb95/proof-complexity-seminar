% Vorlage f�r Seminar-Ausarbeitungen
%
% Dateiname: Vorlage.tex
% zuletzt ge�ndert: 11. Februar 2016
% Autorin: Nicole Schweikardt

% Definition der Dokument-Klasse, in diesem Falle die Klasse SeminarAusarbeitung,
% die in der Datei SeminarAusarbeitung.cls bereitgestellt wird
\documentclass{SeminarAusarbeitung}

\title{Clause space vs. width in resolution}
\author{Narek Bojikian}
\seminar{Beweiskomplexit�t}
\semester{Sommersemester 2020}
\leitung{Prof.\ Dr.\ Christoph Berkholz} 
\institut{Institut f�r Informatik}
\universitaet{Humboldt-Universit�t zu Berlin}

\usepackage{bbold}
\usepackage{xcolor}
\usepackage{graphicx}
\usepackage{float}
\usepackage{natbib}
\usepackage{caption}
\usepackage{tikz}

\begin{document}

\maketitle
\begin{abstract}
	In this report we summarize our presentation of the paper "From Small Space to Small Width
	in Resolution" by Yuval Filmus, et al. \cite{DBLP:journals/tocl/FilmusLMNV15}. We focus on
	the first part of the paper, where the authors provide a constructive proof that resolution
	refutations of low clause space imply refutations of low width. We end with a brief
	go-through of the results in the second part of the paper.
\end{abstract}

\section{Introduction}
Proof systems have been prominent in computer science, as a tool to solve the satisfiability problem
and as a method to prove unsatisfiability of boolean formulas. One of the most popular proof systems
is resolution. It refutes CNF-formulas and it is the base of the DPLL-Algorithm. It is well-studied
in theoretical computer science. For example, exponential lower-bounds have been proved for the
length of resolution refutations of the pigeonhole principle [todo]cite{}. Along side the length of
the proof, other complexity measurements of such proofs have been introduced and well studied. Among
others are width, depth and clause space.

In the paper of our interest, the authors show that a refutation of low clause space yields a
refutation of low width. A result that had been proven by A. Atserias and V. Dalmau in 2007
[todo]cite{}. In the current paper however, the authors present a constructive proof by manipulating
a given refutation of low space to achieve a refutation of low width. They also generalize the proof
to the $r$-DNF proof system. In the second part of the paper, the authors use this method to prove
lower-bounds on the space of refuting Tseitin formulas -a class of formulas that we define later in
this report- using well-known lower-bounds on the width of these formulas. As in the presentation,
we focus on the first part of the paper presenting the techniques developed by the authors. We
finish by a short summary of the results of the second part.  On a side note, the authors also
present in the third part of the paper an analysis of the polynomial calculus proof system, where
they show that similar results are not probably easy to achieve in this system. We skip this part in
our summary, since no major results were proven there, in order to keep more space to the main
results of the paper.

\section{Preliminaries and notation}
We begin this section with formal definitions of formulas, normal forms and assignments. We follow
that by defining the previously mentioned proof systems.
\subsection{Boolean formulas and assignments}
\begin{itemize}
	\item A Boolean formula $\varphi$ is a Boolean combination of a given
		set of variables $\operatorname{Var}(\varphi) := x_1, \dots, x_n$ and
		constants $\mathbb{0}$ and $\mathbb{1}$.
	\item An assignment $\tau$ of $\varphi$ is a Boolean function that assigns to
		each variable of the formula a Boolean value
		$$\tau:\operatorname{Var}(\varphi) \rightarrow \{\mathbb{0,
		1}\}.$$ We write $\tau \models \varphi$ for $\tau$ satisfies
		$\varphi$.
	\item An assignment $\tau$ satisfies a given formula $\varphi$ if and only
		if $[\varphi]^{\tau} = \mathbb{1}$.
	\item A formula is in the negation normal form (\textbf{NNF}), if it
		contains only disjunctions and conjunctions of positive or
		negative literals.
	\item A formula is in the conjunctive normal form (\textbf{CNF}), if it
		is a conjunction of one or more clauses, where each clause is a
		disjunction of one or more literals.
	\item Given a formula $\varphi$ along with an assignment $\tau$ for
		$\varphi$. Let $V \subseteq \operatorname{Var} (\varphi)$ be a
		subset of variables. We define the partial assignment
		$\tau_{|V}$ as the restriction of $\tau$ to $V$, i.e.
		$$\tau_{|V}:V\rightarrow \{\mathbb{0}, \mathbb{1}\}:x\mapsto
		\tau(x).$$
		
	\item[-] For a formula $\varphi$ in CNF and a partial assignment
		$\tau_{|V}$, we say that $\tau_{|V}$ satisfies $\varphi$ if for
		each clause $C$ in $\varphi$, there is a literal $a \in C$ where
		$tau_{|V} \models a$.
\end{itemize}


\subsection{Resolution proof system and r-DNF}




\section{Conclusion}
In conclusion, we have seen two methods to approach the \#SAT problem. We have
seen that the method that had been used frequently to prove efficient running
time in some classes of graphs was inefficient for our targeted class with a
lower-bound on the size of the resulting circuit. On the other hand, we have
also seen that the practical method can turn efficient in theory if a right
elimination order was chosen. One related open question studied in the given
paper is whether all classes, where the theoretical method proved efficient, are
also efficient using the practical Method DPLL by finding a corresponding
elimination ordering and a suitable caching method. Note that the elimination
ordering was natural in the $\beta$-acyclic formulas using a reversed
$\beta$-elimination ordering.

\bibliographystyle{plain}
\bibliography{ref.bib}
\end{document}

